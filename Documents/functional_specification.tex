\documentclass{article}

\usepackage[T1]{fontenc}
\usepackage[utf8]{inputenc}

\title{Specyfikacja funkcjonalna - Poker Royal Flush}
\author{Artur Prasuła}

\begin{document}
\maketitle

\section{Opis ogólny}
    \subsection{Nazwa programu}
        Poker Royal Flush
    
    \subsection{Poruszany problem}
        Program umożliwiać będzie sieciową grę w pokera, pomiędzy sześcioma graczami.
    
    \subsection{Użytkownik docelowy}
        Granie w tę grę, nie będzie wiązało się z ryzykiem przegrania pieniędzy, dzięki czemu program przeznaczony jest do dwóch grup osób.
        Pierwsza grupa to miłośnicy pokera, którzy nie chcą wydawać pieniędzy i martwić się stratami, a
        druga grupa to osoby początkujące, które dopiero uczą się gry i nie chcą przegrywać pieniędzy w swoich pierwszych rozgrywkach.

\section{Opis funkcjonalności}
    \subsection{Jak korzystać z programu}
        Aplikację kliencką obsłużyć będzie można za pomocą myszy komputerowej.
        Po uruchomieniu programu, poprosi on użytkownika o podanie odpowiednich informacji, takich jak: IP serwera, port.
        W czasie rozgrywki, wszystkie możliwe ruchy będą wyświetlone w dolnym menu, a wykonanie ruchu będzie się wiązało z kliknięciem odpowiedniego przycisku w tym menu. Odpowiedni kolor graczy, będzie sygnalizować o tym kogo jest kolej na wykonanie ruchu.
        \\
        Aplikację serwerową należy jedynie uruchomić z odpowiednimi parametrami. W czasie rozgrywki serwer nie będzie wymagać żadnej ingerencji użytkownika.
    
    \subsection{Uruchomienie programu}
        Aplikację kliencką uruchomimy z poziomu GUI systemu operacyjnego, klikając dwa razy na aplikację \textbf{PokerRoyalFlush.jar}.
        \\
        Aplikację serwerową uruchomimy z poziomu linii poleceń.
        \begin{center}
            java -jar PokerRoyalFlushServer.jar [port] [start money] [blind]
        \end{center}
        gdzie,\\
        \textbf{port} - port serwera (domyślnie - 5000)\\
        \textbf{start money} - wartość początkowej ilości pieniędzy jaką dostaje każdy z gracz po dołączeniu do stołu (domyślnie - 200)\\
        \textbf{blind} - wartość ciemnej (domyślnie - 1)
    
    \subsection{Możliwości programu}
        \begin{itemize}
            \item Jednoczesna rozgrywka w pokera dla maksymalnie sześciu graczy przez internet
            \item Automatyczne obliczanie siły układu
            \item Automatyczna kontrola rozgrywki przez serwer
            \item Komunikacja program-użytkownik za pomocą GUI
            \item System odzyskiwania danych przy zerwaniu połączenia
        \end{itemize}

\section{Format danych i struktura plików}
    \subsection{Struktura katalogów}
        Użytkownik docelowy otrzyma jeden katalog w którym będą znajdować się jedynie dwie aplikacje: \textbf{PokerRoyalFlush.jar} - klient gry, \textbf{PokerRoyalFlushServer.jar} - serwer gry.
    
    \subsection{Dane wejściowe}
        \subsubsection{Klient}
            Danymi wejściowymi dla aplikacji klienckiej są informacje potrzebne do połączenia się z serwerem.
            Użytkownikowi po włączeniu programu, ukaże się małe okno w którym, będzie poproszony o podanie: \textbf{IP serwera}, \textbf{port serwera}.
        
        \subsubsection{Serwer}
            Jedyne dane wejściowe jakich serwer potrzebuje to dane, które użytkownik podaje jako parametry uruchomienne. Zostały one opisane w rozdziale \textbf{2.2 Uruchomienie programu}.
    
    \subsection{Dane wyjściowe}
        \subsubsection{Klient}
            Program nie będzie generować ani modyfikować żadnych plików. Wszystkie dane wyjściowe będą pokazywane w GUI.
        
        \subsubsection{Serwer}
            Serwer będzie tworzyć plik logujący \textbf{log.txt}. Wszystkie komunikaty (także te o błędach) będą wypisywane w oknie konsoli oraz do pliku \textbf{log.txt}.

\section{Scenariusz działania programu}
    \subsection{Scenariusz ogólny}
        \begin{enumerate}
            \item Uruchomienie aplikacji serwerowej
            \item Uruchomienie aplikacji klienckiej
            \item Połączenie się aplikacji klienckiej z serwerową
            \item Rozpoczęcie rozgrywki
            \item % DO DOKONCZENIA
        \end{enumerate}
    
    \subsection{Scenariusz szczegółowy}
    
    \subsection{Ekran działania programu}

\section{Testowanie}


\end{document}
