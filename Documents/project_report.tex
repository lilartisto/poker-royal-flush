\documentclass{article}

\usepackage[T1]{fontenc}
\usepackage[utf8]{inputenc}
\usepackage{graphicx}
\usepackage{polski}
\usepackage{listings}
\usepackage{xcolor}

\colorlet{punct}{red!60!black}
\definecolor{background}{HTML}{EEEEEE}
\definecolor{delim}{RGB}{20,105,176}
\colorlet{numb}{magenta!60!black}

\newcommand{\mparagraph}[1]{\paragraph{#1}\mbox{}\vspace{2mm}\\}

\lstdefinelanguage{json}{
    basicstyle=\normalfont\ttfamily,
    numbers=left,
    numberstyle=\scriptsize,
    stepnumber=1,
    numbersep=8pt,
    showstringspaces=false,
    breaklines=true,
    frame=lines,
    literate=
     *{0}{{{\color{numb}0}}}{1}
      {1}{{{\color{numb}1}}}{1}
      {2}{{{\color{numb}2}}}{1}
      {3}{{{\color{numb}3}}}{1}
      {4}{{{\color{numb}4}}}{1}
      {5}{{{\color{numb}5}}}{1}
      {6}{{{\color{numb}6}}}{1}
      {7}{{{\color{numb}7}}}{1}
      {8}{{{\color{numb}8}}}{1}
      {9}{{{\color{numb}9}}}{1}
      {:}{{{\color{punct}{:}}}}{1}
      {,}{{{\color{punct}{,}}}}{1}
      {int}{{{\color{magenta}{int}}}}{3}
      {string}{{{\color{magenta}{string}}}}{6}
      {boolean}{{{\color{magenta}{boolean}}}}{7}
      {"}{{{\color{numb}{"}}}}{1}
      {\{}{{{\color{delim}{\{}}}}{1}
      {\}}{{{\color{delim}{\}}}}}{1}
      {[}{{{\color{delim}{[}}}}{1}
      {]}{{{\color{delim}{]}}}}{1},
}

\title{Sprawozdanie z projektu - Poker Royal Flush}
\author{Artur Prasuła}
\date{}

\begin{document}
\maketitle
\vspace{10mm}
\tableofcontents
\newpage


\section{Opis tematyki projektu}
    Moim zadaniem było stworzenie systemu do gry w pokera.
    System ten najpierw został przeze mnie zaprojektowany, a następnie zaimplementowany w języku Java.
    W skład tego systemu wchodziła aplikacja kliencka oraz aplikacja serwerowa.
    
    \subsection{Klient}
        Klient działa jako aplikacja desktopowa na komputery PC.
        Został on wykonany przy użyciu technologii \textbf{JavaFX}.
        Zastosowany został tzw. lekki klient, czyli klient z jak najmniejszą liczbą zadań.
        Aplikacja ta służy jako interfejs komunikacji pomiędzy graczem, a serwerem.
    
    \subsection{Serwer}
        Serwer został zaimplementowany w języku \textbf{Java}.
        Aplikacja serwerowa pozwala na połączenie z bazą danych, dzięki czemu stan graczy jest zapisany w pamięci pomimo, że gracze nie są dostępni na serwerze.
        Jest to funkcja dodatkowa i serwer działa poprawnie nawet jeśli nie udało się połączyć z bazą danych.

\section{Sposób użycia}

    \subsection{Klient}
        Klient jest aplikacją wyposażoną w graficzny interfejs użytkownika.
        Uruchamiamy go jak każdy inny taki program - na przykład klikając podwójnie w ikonę aplikacji.
        
        \begin{center}
            \includegraphics[width=75mm]{finished_gui_start.png}
            \\
            Ekran startowy aplikacji klienckiej
        \end{center}
        
        W tym miejscu podajemy dane potrzebne do połączenia się z serwerem.
        Po naciśnięciu przycisku ,,Connect'' aplikacja wykona próbę połączenia się.
        Jeśli ta próba zakończy się niepowodzeniem zostanie wyświetlony odpowiedni komunikat.
        W przeciwnym wypadku, aplikacja zmieni scenę na scenę główną - tą z widokiem stołu.
        
        \vspace{4mm}
        
        \begin{center}
            \includegraphics[width=\textwidth]{finished_gui_table.png}
            \\
            Ekran główny aplikacji klienckiej
        \end{center}
    
        Ekran główny służy do brania udziału w rozgrywce.
        To na tym ekranie będziemy widzieć jej stan.
        Ta scena również służy graczowi do wykonywania ruchów.
        Odpowiednie przyciski odblokowują się w sytuacji gdy gracz może wykonać dany ruch.
       
    
    \subsection{Serwer}
        Serwer nie jest wyposażony w graficzny interfejs użytkownika gdyż nie jest on potrzebny.
        Aplikacja po uruchomieniu nie wymaga żadnej ingerencji.
        Bardzo ważna jest natomiast konfiguracja startowa aplikacji.
        W skład tej konfiguracji wchodzą: parametry uruchomienne serwera, konfiguracja połączenia z bazą danych.\\
        \\
        \textbf{Parametry uruchomienne}
            \begin{center}
                java -jar PokerRoyalFlushServer.jar [port] [start money] [blind]
            \end{center}
            gdzie,\\
            \textbf{port} - port serwera (domyślnie - 5000)\\
            \textbf{start money} - wartość początkowej ilości pieniędzy jaką dostaje każdy z gracz po dołączeniu do stołu (domyślnie - 200)\\
            \textbf{blind} - wartość ciemnej (domyślnie - 1)\\
        \\
        \\
        \textbf{Konfiguracja połączenia z bazą danych}\\
            Do skonfigurowania połączenia serwera mojej aplikacji z serwerem bazy danych wykorzystywany jest plik konfiguracyjny \textbf{database.config} w którym definiujemy wszystkie potrzebne do połączenia dane.\\
            Plik \textbf{database.config}:
            
            \begin{lstlisting}[language=json, firstnumber=1]
database.url=url
database.username=login
database.password=haslo
            \end{lstlisting}
            \vspace{2mm}
            
            Jeżeli plik konfiguracyjny będzie zawierać jakieś błędy lub taki plik nie będzie istnieć aplikacja poinformuje użytkownika o błędzie.
            Implementacja serwera przewiduje w takiej sytuacji dalsze działanie serwera z nieaktywną funkcjonalnością bazy danych.


\section{Testowanie programu}
    Poszczególne klasy w programie testowane były za pomocą testów jednostkowych.
    Do tego celu użyłem narzędzia \textbf{JUnit}.
    Integracja poszczególnych funkcjonalności i GUI były testowane ręcznie.\\
    Tworzenie i wykonywanie wyżej wymienionych testów było nieodłącznym elementem mojej pracy i poświęcałem im dużą uwagę w całym okresie trwania prac przy kodzie projektu.\\
    \\
    Problematyczne okazało się testowanie funkcjonalności, które do poprawnego działania wykorzystywały odpowiedź od drugiej aplikacji.
    To znaczy na przykład funkcje po stronie serwera, które wykorzystywały komunikaty od klienta.
    Wydaje mi się, że tą problematyczność można by rozwiązać refraktoryzując kod.
    

\section{Zmiany w projekcie względem specyfikacji implementacyjnej}
    Podczas prac projektowania systemu starałem się, aby zaplanować jak najbardziej szczegółowo jak aplikacja będzie działać.
    Dokładniej rzecz ujmując:
        jak będzie wyglądać komunikacja klient-serwer,
        jak wyglądać będzie współpraca poszczególnych modułów w aplikacji klienckiej i serwerowej 
        oraz jak wyglądać będą poszczególne klasy w programie.\\
    \\
    Uważam, że faza projektowania przebiegła z pozytywnym skutkiem i końcowy projekt nie różni się dużo od tego zaprojektowanego na papierze.
    Niestety w trakcie implementacji okazało się, że system wymaga małych poprawek, które oczywiście zrealizowałem.
    
    \newpage
    \paragraph{Lista zmian w projekcie względem specyfikacji implementacyjnej}
    \begin{itemize}
        \item dodatkowe komunikaty wysyłane przez klienta do serwera
        \begin{itemize}
            \item komunikat \textbf{connect} - informuje o chęci dołączenia do serwera
            \item komunikat \textbf{disconnect} - informuje o rozłączeniu się klienta od serwera
        \end{itemize}
        \item dodatkowe komunikaty wysyłane przez serwer do klienta
        \begin{itemize}
            \item komunikat \textbf{connect} - odpowiedź informująca o tym czy klient dołączył do serwera
        \end{itemize}
        \item dodatkowe klasy interpretujące i formatujące wyżej wymienione komunikaty
        \item rozbicie klasy \textbf{server.communication.MsgFormat} na poszczególne klasy odpowiadające danym komunikatom
        \item dodatkowa klasa \textbf{server.communuication.PlayerListener} - jej zadaniem jest nasłuchiwanie na wiadomości przysyłane przez klientów (wcześniej ta funkcjonalność była zawarta w klasie \textbf{ClientConnector})
        \item zastosowanie wzorca projektowego \textbf{singleton} w niektórych klasach
        \item usunięcie klasy \textbf{poker.client.Game} - przyczyną jest zastosowanie wyżej wymienionego wzorca projektowego
        \item dodanie pliku \textbf{style.css} aby aplikacja kliencka ładniej się prezentowała
        \item zmiany w polach i metodach w niektórych klasach
    \end{itemize}


\section{Wnioski i podsumowanie projektu}
    Projekt ten jest moim największym projektem, który napisałem.
    Podejmując się go, byłem pełny wątpliwości czy poradzę sobie z tym zadaniem.
    Przez to, że nie stworzyłem wcześniej żadnego poważnego systemu z architekturą klient-serwer i integracją aplikacji z bazą danych moje wątpliwości były jeszcze większe.
    Patrząc teraz, na skończony już projekt uważam, że jako niedoświadczony programista poradziłem sobie z tym zadaniem zaskakująco dobrze.
    
    \subsection{Co dało się zrobić lepiej?}
        Mimo, że cały system działa w pełni poprawnie, widzę w nim kilka niedociągnięć.
        
        \vspace{3mm}
        \textbf{Nasłuchiwanie na komunikaty od graczy}\\
            Aktualne rozwiązanie nasłuchiwania na komunikaty opiera się na tym, że dla każdego, nowo dołączonego gracza, serwer uruchamia oddzielny wątek, którego zadaniem jest nasłuchiwanie na wiadomości wysyłane przez klienta.
            Jest to rozwiązanie kosztowne dla maszyny na której uruchomiony jest serwer.
            Problem ten można rozwiązać na przykład uruchamiając jeden wątek nasłuchujący, a nowo dołączonych graczy po prostu dołączyć do iteracji nasłuchiwania na komunikaty.
            
        \vspace{3mm}
        \textbf{Czytelność widoku stołu w aplikacji klienckiej}\\
            Widok stołu uważam za czytelny, ale widzę w nim dwa niedociągnięcia.
            Pierwszym z nich jest brak jakiejś kontrolki, która informowała by o tym ile czasu zostało graczowi na wykonanie danego ruchu.
            \\
            Drugim problemem jest czytelność stanu gracza - czy jest w grze, czy jest jego kolej na ruch.
            Aktualnie stan gracza wyświetlany jest jako kolor ramki tego gracza. Jest to rozwiązanie wystarczające, ale dla nowych graczy takie rozwiązanie może być mylące.
        
    
    \subsection{Z czego jestem zadowolony?}
        Widzę, że mój projekt nie jest bezbłędny, ale wciąż jestem zadowolony z jego ogółu.
        W szczególności jestem zadowolony z kilku ważnych decyzji podjętych w fazie projektowania.
        
        \vspace{3mm}
        \textbf{Komunikaty wysyłane pomiędzy klientem a serwerem}\\
            Podczas projektowania tego elementu programu, postawiłem na jak największą prostotę tej komunikacji.
            Dlatego liczba wszystkich komunikatów jest bardzo mała, ale dzięki uniwersalności komunikatu \textbf{info}, system nie wymaga większej ich ilości.
        
        \vspace{3mm}
        \textbf{System punktów siły układu kart}\\
            Tworząc funkcjonalność odpowiadającą za wyznaczanie zwycięzcy spośród graczy wpadłem na pomysł systemu punktów.
            Punkty w tym systemie odzwierciedlałyby siłę danego układu.
            Zasada działania jest bardzo prosta:\\
            Jeśli układ A jest silniejszy od układu B, to metoda PokerHandCalculator.handStrength() spełnia zależność:
            \begin{center}
                handStrength(A) > handStrength(B)
            \end{center} 
            Jeśli układ A jest tak samo silny jak układ B, to metoda PokerHandCalculator.handStrength() spełnia zależność:
            \begin{center}
                handStrength(A) = handStrength(B)
            \end{center}
            Zadanie to nie było proste.
            Liczba wszystkich możliwych układów w tej grze karcianej wynosi 2 598 960.
            
\end{document}